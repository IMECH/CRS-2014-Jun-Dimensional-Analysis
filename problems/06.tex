\begin{problem}[06]
从量纲幂次式的讨论中得到的偏导数关系, 求出量纲函数的最终表示式.
\end{problem}
% --------------------------------------------------------------------
\begin{solution}
书\cite{tan_dimensional_2011}中式(2.10)为从量纲幂次式的讨论中得到的偏导数关系, 即以下三式:
\[
\frac{\frac{\partial f}{\partial r_l}(r_l, r_m, r_t)}{f(r_l, r_m, r_t)} = \frac{\alpha}{r_l}
, \qquad
\frac{\frac{\partial f}{\partial r_m}(r_l, r_m, r_t)}{f(r_l, r_m, r_t)} = \frac{\beta}{r_m}
, \qquad
\frac{\frac{\partial f}{\partial r_t}(r_l, r_m, r_t)}{f(r_l, r_m, r_t)} = \frac{\gamma}{r_t}
\]
整理上式, 并将$f(r_l, r_m, r_t)$简写为$f$可得
\begin{equation}\label{eq:partialF/partialR}
\frac{\partial f}{\partial r_l}= \frac{\alpha}{r_l}f
, \qquad
\frac{\partial f}{\partial r_m}= \frac{\beta}{r_m}f
, \qquad
\frac{\partial f}{\partial r_t}= \frac{\gamma}{r_t}f
\end{equation}
对函数$f(r_l, r_m, r_t)$进行全微分得
\begin{equation}\label{eq:df}
df = \frac{\partial f}{\partial r_l}dr_l + \frac{\partial f}{\partial r_m}dr_m + \frac{\partial f}{\partial r_t}dr_t
\end{equation}
将式(\ref{eq:partialF/partialR})代入式(\ref{eq:df})得
\begin{equation}
df = \alpha f \frac{dr_l}{r_l} + \beta f \frac{dr_m}{r_m} + \gamma f \frac{dr_t}{r_t} 
\quad \Longrightarrow \quad 
\frac{df}{f} = \alpha \frac{dr_l}{r_l} + \beta \frac{dr_m}{r_m} + \gamma \frac{dr_t}{r_t}
\end{equation}
对上式得到的关系两边同时进行积分得
\begin{equation}
\ln f = \alpha \ln r_l + \beta \ln r_m + \gamma \ln r_t + \ln C
\quad \Longrightarrow \quad 
f = C r_r^{\alpha} r_m^{\beta} r_t^{\gamma}
\end{equation}
其中$C$为待定常数, 若单位系中的长度, 质量, 和时间不变, 则对应的缩放倍数可看成$r_l=1$, $r_m=1$, $r_t=1$, 显然有$f(1,1,1) = C = 1$. 因此
\begin{equation}
f = r_r^{\alpha} r_m^{\beta} r_t^{\gamma}
\end{equation}
可见, 与之相对应, 物理量$X$的量纲表达式应为
\[
[X] = L^\alpha M^\beta T^\gamma
\]
\end{solution}
