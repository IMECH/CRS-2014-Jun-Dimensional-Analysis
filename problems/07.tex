\begin{problem}[07]
查阅基尔比契夫提出的``相似三定理''说的是什么? 它与$\Pi$定理的说法不同, 哪种说法更为本质?
\end{problem}
% --------------------------------------------------------------------
\begin{solution}
基尔比契夫提出的``相似三定理''可分别表述如下\cite{zlvo3}:
\begin{itemize}
\item \textbf{相似第一定理}: 彼此相似的现象, 其同名相似准则的数值相同.
\item \textbf{相似第二定理}: 物理现象中各物理量之间的关系, 可以化为各相似准则之间的关系.
\item \textbf{相似第三定理}: 如果两个现象的单值条件相似, 而且由单值量组成的同名相似准则数值相同, 则这两个现象相似.
\end{itemize}
而$\Pi$定理可概括为:
\begin{itemize}
\item \textbf{$\Pi$定理}: 问题中若有$N$个变量, 即$n$个自变量$a_1,a_2,\cdots a_n$和应变量$a$, 那么因变量$a$可表示为自变量的函数$a=f(a_1,a_2,\cdots, a_n)$, 而基本量的数目是$k$, 那么一定形成$N-k$个无量纲量, 即1个无量纲应变量$\Pi$和$n-k$个无量纲自变量$\Pi_1,\Pi_2,\cdots, \Pi_{n-k}$, 它们之间形成确定的函数关系$\Pi = F(\Pi_1,\Pi_2,\cdots, \Pi_{n-k})$.
\end{itemize}
基尔比契夫提出的``相似三定理''与$\Pi$定理相比, 显然$\Pi$定理更为简单, 两者表述中相对应的关系如下:
\begin{itemize}
\item $\Pi$定理中的``变量''对应相似三定理表述中的``物理量'';
\item $\Pi$定理中的``无量纲量''对应相似三定理表述中的``同名相似准则'';
\item 显然$\Pi$定理中, 对同一个函数$F$, 各无量纲量值相同时, $F$的值也必定唯一, 这对应相似三定理表述中的``单值条件相似''.
\end{itemize}
与相似三定理相比, $\Pi$定理也更为本质. $\Pi$定理包含了相似三定理: 
\begin{itemize}
\item \textbf{相似第一定理}: 若两个物理现象具有相同的函数关系$\Pi = F(\Pi_1,\Pi_2,\cdots, \Pi_{n-k})$, 则其无量纲量数值相同.
\item \textbf{相似第二定理}: 物理现象中的$N=n+1$个变量间的关系$a=f(a_1,a_2,\cdots, a_n)$, 可以化为无量纲量间的函数关系$\Pi = F(\Pi_1,\Pi_2,\cdots, \Pi_{n-k})$.
\item \textbf{相似第三定理}: 若两物理现象中各无量纲量$\Pi_1,\Pi_2,\cdots,\Pi_{N-k}$数值相等, 且函数$F$关系相同, 则$\Pi = F(\Pi_1,\Pi_2,\cdots, \Pi_{n-k})$的函数值也必定单值唯一.
\end{itemize}
\end{solution}
