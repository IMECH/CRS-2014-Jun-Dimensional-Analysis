\begin{problem}[01]
什么是``量纲分析''?
\end{problem}
% --------------------------------------------------------------------
\begin{solution}
量纲分析就是在量纲法则(主要是$\Pi$定理)的原则下, 分析和探求物理量之间的关系. 量纲分析的精神实质:
\begin{itemize}
\item 只有同类量才能比较大小.
\item 物理现像和物理规律与所选用的度量单位无关.
\end{itemize}
具体说来: 量纲分析就是在控制某类物理现象或问题的物理量中, 先定一组物理量作为基本量, 并取作单位系统, 用以度量这类现象或问题中的任何物理量, 这样得到的该物理量的大小数值是无量纲的, 反映无量纲的因变量和自变量之间的因果关系, 也必然客观地反映这类现象的本质.
\end{solution}
