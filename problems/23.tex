\begin{problem}[23]
弹性体中体波的传播有无色散现象, 说说物理原因?
\end{problem}
% --------------------------------------------------------------------
\begin{solution}
波有无色散是指波速是否与波长有关, 对于弹性体中体波的传播有无色散现象的讨论, 分别从量纲分析和物理分析两方面展开:
\begin{itemize}
\item \textbf{量纲分析}: 对于弹性体中体波, 假设波长为$\lambda$, 密度为$\rho$, 杨氏模量为$E$, 泊松比为$\nu$, 体波波速$c$应当是上述4个控制参数的函数
\[
c = f(\rho, E, \nu; \lambda)
\]
上式中各物理量的量纲分别为: $[c]=LT^{-1}$, $[\rho]=ML^{-3}$, $[E]=ML^{-1}T^{-2}$, $\nu$为无量纲量, $[\lambda]=L$. 取$\rho$, $E$, $\lambda$为基本量, 于是有以下无量纲关系:
\[
\frac{c}{\sqrt{E/\rho}} = f(1,1,\nu,1) = f(\nu) 
\quad\Longrightarrow\quad
c = \sqrt{\frac{E}{\rho}}~f(\nu)
\]
可见弹性体中体波的波速与波长$\lambda$无关, 即弹性体中体波的传播无色散现象.

\item \textbf{物理分析}: 体波是指体内一点扰动在有限时间传播到有限体积的波, 在体波的传播中没有特征长度, 也就没有长度单位去度量波长, 所以体波波速的控制参数没有波长$\lambda$. \textbf{因此在上述量纲分析中可直接不引入波长$\lambda$}.
\end{itemize}
\end{solution}
