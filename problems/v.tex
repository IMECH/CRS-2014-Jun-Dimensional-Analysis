\begin{problem}[05]
怎样表征固体的弹塑性, 有何重要特点?
\end{problem}
% --------------------------------------------------------------------
\begin{solution}
在弹性阶段(材料受应力较小), 当外力移除后,物体会恢复成原来形变前的状态. 当外力造成的应力超过一定范围时, 物体会产生不可逆的永久形变, 外力移除后,物体无法完全恢复成原来的状态, 称为弹塑性变形. 固体的弹塑性可由应力-应变曲线及屈服点描述: 
\begin{itemize}
\item 屈服点前是弹性变形, 应力-应变程线性关系, 可逆; 
\item 屈服点后是弹塑性变形, 应力-应变程非线性关系, 不可逆.
\end{itemize}
\end{solution}
