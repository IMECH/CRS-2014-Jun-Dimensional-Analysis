\begin{problem}[20]
什么样的结构物需要考虑重力的作用?
\end{problem}
% --------------------------------------------------------------------
\begin{solution}
设结构物的高度为$h$, 密度为$\rho$, 临界屈服应力$p_\mathrm{cr}$, 重力加速度为$g$, 特征分布载荷为$\Sigma$. 重力所引起的应力大约和$\rho g h$差不多大, 以下情况需要考虑重力的作用:
\begin{itemize}
\item 重力占主导作用或相对于外载并不是很小时, 则显然要考虑重力.
\[
\frac{\rho g h}{\Sigma}\gtrsim 1
\]
如分析悬臂梁在自重作用下的挠度分布(这种情况外载为零).
\item 当结构物的高度$h$较大时, 以至于重力作用超过临界屈服应力或与临界屈服应力相当时, 也需要考虑重力的作用.
\[
\frac{\rho g h}{p_\mathrm{cr}}\gtrsim 1
\]
\end{itemize}
\end{solution}
