\begin{problem}[08]
用隐函数法证明$\Pi$定理.
\end{problem}
% --------------------------------------------------------------------
\begin{solution}
把物理问题的因变量和自变量都统一视作变量, 若其总数为$N$, 并分别记为$a_1,a_2,\cdots,a_N$, 那么物理规律可表示为以下隐函数关系:
\[
f(a_1, a_2, \cdots, a_N) = 0
\]
在上式涉及到的$N$个变量中, 选出$k$个基本量, 不妨排在前面, 它们是$a_1,a_2,\cdots,a_k$, 其量纲分别为$A_1, A_2, \cdots, A_k$. 而后面$N-k$个变量则是导出量, 其量纲可表示为基本量的量纲的幂次式:
\begin{align*}
[a_{k+1}] & =A_{1}^{p_{1}}A_{2}^{p_{2}}\cdots A_{k}^{p_{k}}\\{}
[a_{k+2}] & =A_{1}^{q_{1}}A_{2}^{q_{2}}\cdots A_{k}^{q_{k}}\\
& \vdots\\{}
[a_{N}] & =A_{1}^{t_{1}}A_{2}^{t_{2}}\cdots A_{k}^{t_{k}}
\end{align*}
用$k$个基本量单位系统来度量函数关系中的各变量, 由此得到以下关系
\[
f\Bigg(\underbrace{1,\cdots,1}_{k\text{个}},\frac{a_{k+1}}{a_{1}^{p_{1}}a_{2}^{p_{2}}\cdots a_{k}^{p_{k}}},\; \frac{a_{k+2}}{a_{1}^{q_{1}}a_{2}^{q_{2}}\cdots a_{k}^{q_{k}}},\cdots\frac{a_{N}}{a_{1}^{t_{1}}a_{2}^{t_{2}}\cdots a_{k}^{t_{k}}}\Bigg)=0
\]
上式中对函数$f$起作用的变量仅为后面的$N-k$个无量纲变量, 分别记为$\Pi_1,\Pi_2,\cdots, \Pi_{N-k}$, 则上式可表示为
\[
f(\Pi_1,\Pi_2,\cdots, \Pi_{N-k})=0
\]
\end{solution}
