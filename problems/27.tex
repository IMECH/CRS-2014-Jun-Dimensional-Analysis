\begin{problem}[27]
什么是几何相似? 什么是几何相似律? 举例说明.
\end{problem}
% --------------------------------------------------------------------
\begin{solution}
\begin{itemize}
\item \textbf{几何相似}: 模型和原型的形状几何相似, 即模型和原型在任何几何维度上的长度比值相等, 即
\[
\bigg(\frac{l'}{l}, \frac{l''}{l},\cdots\bigg)_\mathrm{p} = \bigg(\frac{l'}{l}, \frac{l''}{l},\cdots\bigg)_\mathrm{m}
\]
\textbf{举例}: 例如在第\hyperref[problem:12]{12}题``\textcolor{blue}{能否用水洞做机翼的模型实验, 或用风洞做潜艇的模型实验?}''的结论中: 用水洞做机翼的模型实验, 或用风洞做潜艇的模型的必要条件之一就是几何相似:
$(l'/l,\cdots)_\mathrm{m}=(l'/l,\cdots)_\mathrm{p}$ 及 $\alpha_m = \alpha_p$.
\item \textbf{几何相似律}: 若模型采用与原型同样的材料($\alpha_E=1$且$\alpha_\nu=1$), 则在满足几何形状相似和分布函数相同的条件下, 在几何相似点
\[
\bigg(\frac{x}{l},\frac{y}{l},\frac{z}{l}\bigg)_\mathrm{p} 
=
\bigg(\frac{x}{l},\frac{y}{l},\frac{z}{l}\bigg)_\mathrm{m} 
\]
上, 有相同的相对位移, 应力和应变
\[
\bigg(\frac{w^i}{l}\bigg)_\mathrm{p} = \bigg(\frac{w^i}{l}\bigg)_\mathrm{m},~~
(\sigma_{ij})_\mathrm{p} = (\sigma_{ij})_\mathrm{m},~~
(\varepsilon_{ij})_\mathrm{p} = (\varepsilon_{ij})_\mathrm{m}
\]
\textbf{举例}: 例如在第\hyperref[problem:17]{17}题``\textcolor{blue}{讨论两端固定的梁在分布载荷作用下的挠度.}''的结论中: 只要保持$(q_m l^3/(EI))_\mathrm{p}=(q_m l^3/(EI))_\mathrm{m}$及$(l_q/l)_\mathrm{m}(l_q/l)_\mathrm{p}$, 就会有同一个无量纲挠度分布$w=lf(x/l)$.
\end{itemize}
\end{solution}
