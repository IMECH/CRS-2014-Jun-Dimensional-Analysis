%\invisiblesection{近三年考试试题}
\appendix
\appendixpage
%\section{近三年考试试题}

\noindent\textbf{注: 这里给出了近三年的量纲分析考试题, 仅供后届同学平时练习和期末复习参考. 由试题可以看出, 谈老师的考题大多源于课堂和课后习题, 偏重基础和概念. 只要上课认真听讲, 作业认真做, 考试基本没有难度.}

\section{2012试题}
\begin{enumerate}
\item 怎样正确理解所谓``低速绕流''和所谓``高速冲击''的提法?
\item 请对单摆的周期做出正确又简炼的量纲分析.
\item 在两种液体中已知一种液体的粘性系数, 如何用一个小球来测定另一种液体的粘性系数?
\item 弹性波有哪几类? 对每一类弹性简谐波, 从物理上考虑, 可以用什么特征量来度量波长?
\end{enumerate}

\section{2013试题}

\begin{enumerate}
\item 对固体力学问题和流体力学问题, 各举两个无量纲数, 并说明将该数用在判据中的例子.
\item 物理问题中的因果关系是否应当由无量纲的形式? 说明理由, 并举例.
\item 对圆管流动来说, 决定总管阻和决定摩擦系数的自变量是否相同? 为什么?
\item 色散波和非色散波各举一例, 并说明发生色散现象的物理原因.
\end{enumerate}

\section{2014试题}
\begin{enumerate}
\item 一个家庭主妇和一个总理怎么样用量纲分析来管家和治国?
\item 对变形问题和流动问题分别给出两个无量纲数, 并说明物理意义和判据中的应用.
\item 说明用小球自由落体运动测定液体的粘性系数的原理.
\item 弹性波有哪几类? 从物理上考虑, 用什么量来度量简谐波的波长?
\end{enumerate}
